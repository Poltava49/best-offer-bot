\documentclass[14pt, a4paper]{extarticle} % Базовый класс
\usepackage[T2A]{fontenc} % Кодировка
\usepackage[utf8]{inputenc} % Кодировка
\usepackage[russian]{babel} % Русский язык
\usepackage{geometry} % Поля
\geometry{left=3cm, right=1.5cm, top=2cm, bottom=2cm}
\usepackage{titlesec} % Стиль заголовков
\usepackage{enumitem} % Списки
\usepackage{hyperref} % Ссылки

\titleformat{\section}[block]{\normalfont\Large\bfseries}{\thesection}{1em}{}
\titleformat{\subsection}[block]{\normalfont\large\bfseries}{\thesubsection}{1em}{}

\begin{document}

\begin{center}
    \Large\textbf{ТЕХНИЧЕСКОЕ ЗАДАНИЕ} \\
    \vspace{0.5cm}
    \normalsize на разработку сервиса для анализа цен и ассортимента на маркетплейсах РФ
\end{center}

\section*{1. Общие сведения}
\textbf{1.1. Наименование проекта:} Parser_telegram_bot — сервис парсинга данных о товарах, анализа цен и ассортимента в Telegram.
\textbf{1.2. Контекст:} Пользователям сложно и трудозатратно самостоятельно проанализировать текущие товарные предложения на разных маркетплейсах. Данный сервис автоматизирует сбор данных о товарах, анализ  цен и  итоговые релевантные предложения через удобный Telegram-интерфейс.
\textbf{1.3. Используемые технологии (стек):}
\begin{itemize}[noitemsep, topsep=0pt]
    \item \textbf{Backend (Парсинг \& API):} Python, FastAPI, библиотеки для парсинга (BeautifulSoup4, Scrapy, Selenium/Playwright - по необходимости).
    \item \textbf{Работа с данными и задачи по расписанию:} PostgreSQL (данные о пользователях, хранение запросов и итоговые выданные результаты в абсолютах (цена, рейтинг, бренды).
    \item \textbf{Внешний интерфейс:} Telegram Bot (библиотека python-telegram-bot).
    \item \textbf{Инфраструктура:} Docker, Docker Compose.
\end{itemize}

\section*{2. Цели и задачи проекта}
\subsection*{2.1. Цели}
\begin{enumerate}
    \item Создать автоматизированный сервис для анализа цен на товары с выбранных маркетплейсов. Сервис находит релевантное товарное предложение с самой низкой ценой или лучшим соотношением "цена/рейтинг", информирует о сроках доставки, сравнивает цены, что дает полную картину для выбора.



\end{enumerate}
\subsection*{2.2. Задачи}
\begin{enumerate}
    \item Разработать модули парсеров для целевых маркетплейсов (Wildberries, OZON).
    \item Спроектировать и реализовать базу данных для хранения информации о авторизованных пользователях, истории запросов и итоговых данных в абсолютах (средняя цена, рейтинг, доставка).
    \item Обеспечить базовое логирование и обработку ошибок.
\end{enumerate}

\section*{3. Требования к системе}
\subsection*{3.1. Функциональные требования (ФТ)}
\begin{enumerate}
    \item \textbf{ФТ-01. Базовый запрос на поиск и анализ товара:}
    \begin{itemize}
        \item Пользователь выбирает один или все предложенные площадки по продаже товаров для анализа
        \item Пользователь вводит наименование товара или категорию товара для поиска по множеству брендов в рамках категории
    \end{itemize}
    \item \textbf{ФТ-02. Парсинг данных:}
    \begin{itemize}
        \item Система должна извлекать из страницы товара: наименование, текущую цену, старую цену (если есть), рейтинг, наличие, срок доставки, кол-во отзывов.
    \end{itemize}
    \item \textbf{ФТ-03. Анализ:}
    \begin{itemize}
        \item Система выдает результирующим ответом анализ по выбранных площадках по запрашиваемому товару в разрезе: распределение цены, топ-10 брендов, если запрос на уровне категории, кол-во предложений
        \item Система отправляет пользователю визуализацию анализа (matplotlib, seaborn)
        \item Система выдает топ-5 лучших предложений с лучшим соотношением "цена/рейтинг" и кол-во отзывов
    \end{itemize}
    \item \textbf{ФТ-04. Администрирование:}
    \begin{itemize}
        \item \textbf{(Опционально)} Существует панель или команды для администратора для просмотра состояния системы, логов, количества пользователей.
    \end{itemize}
\end{enumerate}
\subsection*{3.2. Нефункциональные требования (НФТ)}
\begin{enumerate}
    \item \textbf{НФТ-01. Производительность:} Парсинг одного товара не должен занимать более 1 минуты.
    \item \textbf{НФТ-02. Надежность:} Система должна корректно обрабатывать ошибки парсинга (изменение структуры сайта, недоступность) и не прекращать работу целиком.
    \item \textbf{НФТ-03. Безопасность парсинга:} Парсер должен использовать задержки между запросами и корректные HTTP-заголовки (User-Agent), чтобы минимизировать риск блокировки IP.
    \item \textbf{НФТ-04. Масштабируемость:} Архитектура должна позволять добавлять парсеры для новых маркетплейсов без переписывания основной логики.
\end{enumerate}

\section*{4. Описание системы (Архитектура)}
Система состоит из следующих основных модулей:
\subsection*{4.1. Модуль парсинга (Parser Core)}
\begin{itemize}
    \item Отвечает за сбор данных с маркетплейсов.
    \item Реализуется как набор отдельных ``спайдеров'' (по одному на каждый маркетплейс).
    \item Система выполняет запрос к странице, извлекает структурированные данные и сохраняет в БД с привязкой к пользователю.
\end{itemize}
\subsection*{4.2. База данных (PostgreSQL)}
Хранит:
\begin{itemize}
    \item \textbf{Пользователи (users):} ID Telegram, имя, дата регистрации.
    \item \textbf{Товары (products):} Название, ссылка, текущая цена, дата запроса.
\end{itemize}
\subsection*{4.3. Модуль Telegram-бота (Bot Core)}
\begin{itemize}
    \item Взаимодействует с Telegram API.
    \item Обрабатывает команды (/start, /help, /list) и текстовые сообщения (ссылки).
    \item При получении данных от модуля анализа отправляет пользователям уведомления.
\end{itemize}

\section*{5. План работ (Roadmap)}
\textbf{Фаза 1: Прототип (Недели 1-2)}
\begin{enumerate}
    \item Настройка окружения (Python, виртуальное окружение, Docker).
    \item Разработка простейшего парсера для одного маркетплейса (BeautifulSoup/Requests).
    \item Создание схемы БД и ее реализация.
    \item Написание базовой логики для сохранения данных.
\end{enumerate}
\textbf{Фаза 2: Ядро системы (Недели 3-4)}
\begin{enumerate}
    \item Разработка базового Telegram-бота с командой /start и добавлением возможности множественного выбора площадок для парсинга и анализа.
    \item Привязка пользователей и товаров в БД.
\end{enumerate}
\textbf{Фаза 3: Логика и анализ (Недели 5-6)}
\begin{enumerate}
    \item Реализация логики анализа полученных данных
    \item Реализация логики выдачи итогов анализа
    \item Добавление парсера для второго маркетплейса.
    \item Улучшение обработки ошибок и логирования.
\end{enumerate}
\textbf{Фаза 4: Доработка и развертывание (Недели 7-8)}
\begin{enumerate}
    \item Создание docker-compose.yml для запуска всех сервисов.
    \item Написание базовой документации в README.md.
    \item Тестирование и отладка.
\end{enumerate}

\section*{6. Критерии приемки}
Проект считается успешно завершенным, если:
\begin{enumerate}
    \item Пользователь может через Telegram-бота получить релевантны пул товаров и базовый анализ на выбранном маркетплейсе(-ах) по своему запросу.
    \item Система сохраняет запрос товар и его текущую цену в базу данных.
    \item Пользователь может запросить список своих запросов товаров.
    \item Система может работать непрерывно в фоновом режиме (запущенная через docker-compose up).
\end{enumerate}

\end{document}